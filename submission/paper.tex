%% To submit your paper:
\documentclass[draft,linenumbers]{agujournal}
\draftfalse

%% For final version.
% \documentclass{agujournal}

\journalname{Water Resources Research}

\usepackage{url, bbm, comment}

\begin{document}


\title{I don't know, are you sure we want to do this?\\
Sea level adaptation decisions under uncertainty}

\authors{T. L. Thorarinsdottir\affil{1}, P. Guttorp\affil{1}, M. Drews\affil{2}, and K. de Bruin\affil{3,4}}

\affiliation{1}{Norwegian Computing Center, Oslo, Norway}
\affiliation{2}{Technical University of Denmark, Copenhagen, Denmark}
\affiliation{3}{Center for International Climate and Environmental Research, Oslo, Norway}
\affiliation{4}{Wageningen Environmental Research, Wageningen, The Netherlands}

\correspondingauthor{T. L. Thorarinsdottir}{thordis@nr.no}

\begin{keypoints}
\item Decisions on adaptation measures need to take careful account of uncertainties
\item More careful models are needed for increased costs of the effects of sea level rise
\item Modeling local sea level rise is essential
\end{keypoints}


\begin{abstract}
Sea level rise has serious consequences for harbor infrastructure, storm drains and sewer systems, and many other issues. Adapting to sea level rise requires comparing different possible adaptation strategies, comparing the cost of different actions (including no action), and assessing at what point in time the chosen strategy should be implemented. All these decisions must be made under considerable uncertainty--in the amount of sea level rise, in the cost of adaptation actions, and in the cost of no action. We develop two illustrative examples: for Bergen on Norway's west coast and for Esbjerg on the west coast of Denmark. Different components of uncertainty are visualized. We show that failing to take uncertainty into account can increase the projected damage costs by an order of magnitude.
\end{abstract}


%% ------------------------------------------------------------------------ %%
%  BEGIN ARTICLE
%% ------------------------------------------------------------------------ %%


\section{Introduction}\label{sec:intro}
Long-term planning and decision-making regarding societal infrastructure along coastal areas must account for a changing climate. In particular, potential changes in local sea level can yield numerous effects such as frequent flooding, inundation and back flow of storm drainage and sewer systems, destructive erosion and contamination of wetlands and other habitats. Considerable challenges continue to exist in understanding  the impacts of a combination of sea level rise and storm surges and the implementation of potential adaptation options to reduce these impacts. Severe inherent uncertainties persist along all parts of the processing chain from climate projections to adaptation assessments, and it is critical that decision-making appropriately accounts for this.
 
As adaptation decision-making is an ongoing process of weighing and choosing which measures should be taken at which moment in time \citep{Hallegatte&2012} adaptive planning methods need to support decisions in the short term, while considering long-term developments. Challenges of adaptation decision-making under uncertainty relate to the incorporation of spatial, inter-temporal and flexibility aspects of adaptation priorities \citep{FankhauserSoare2013}, and the linkage with specific characteristics of sectors and contexts \citep{BisaroSwartHinkel2016, HinkelBisaro2016}. Several economic decision support tools and methods exist for adaptation assessment under uncertainty \citep[e.g.][]{Chambwera&2014, WilbyDessai2010, WalkerHaasnootKwakkel2013}. However, \citet{Watkiss&2015} conclude that these tools are very resource intensive and complex in the context of long-term adaptation investment decisions and call for the development of ``light touch'' approaches to better support local adaptation making.
 
In this paper we combine the uncertainty of sea level projections with a simplified decision framework under uncertainty where planners want to know how early they should implement costly adaptation measures.  We discuss the advantages of using a fully probabilistic approach in a light touch decision framework that accounts for uncertainty both in climate projections and damage assessments, where information on uncertainty is turned into a driver of change rather than a barrier.
 
We illustrate the sea level rise projections and timing of adaptation measures for two case studies in Northern Europe, namely the city of Bergen in Norway and Esbjerg in Denmark. The Norwegian city of Bergen is the capital of Hordaland County. The city center is located on Byfjorden, and is surrounded by mountains. It has the largest port in Norway, both in terms of freight and passengers. The historic harbor area, Bryggen, is the only Hanseatic trade center remaining in its original style, and has been declared a UN-ESCO World Heritage site\footnote{See \url{http://whc.unesco.org/en/list/59}.}. Bryggen is regularly flooded at extreme tides, and it is feared that as sea levels rise, floods will become a major problem in other parts of Bergen as well \citep{bergenreport}.

The municipality of Bergen has, in cooperation with private actors, analyzed several possible adaptation measures against sea level rise. The measures range from an outer barrier that would protect the entire metropolitan area to various protection measures of limited areas in the inner harbor \citep{bergenreport}. While the viability of the constructions and the associated construction costs have been carefully analyzed, the optimal timing of potential adaptation measures and the effects of the associated uncertainties have yet to be investigated. We perform such an analysis where we consider uncertainty in projected sea level rise, damage costs and the effects of sea level rise on changes in damage costs. 

Esbjerg, on the west coast of Jutland, has the second largest harbor in Denmark. It can be subjected to substantial storm surges, which, in conjunction with heavy rainfall, can flood substantial parts of the town.
 
The remainder of the paper is organized as follows. Section \ref{sealevelproj} describes our approach to projecting sea level. In section \ref{decision_tools_PartI} we describe the decision framework focused on the timing of proactive adaptation decision. We apply this framework to sea level projections and adaptation for Bergen, Norway, and for Esbjerg, Denmark in section \ref{cases}. In the final section \ref{unc} we demonstrate the consequences of ignoring the uncertainty in the projections.

\section{Sea level projections}
\label{sealevelproj}

We project local sea level changes by Modeling two processes:
\begin{enumerate}
\item The relationship between global temperature and global sea level;
\item The relationship between global sea level and local sea level.
\end{enumerate}

\subsection{Global sea level}
Most climate models do not explicitly provide sea level as an output of the calculations. Rather, the IPCC AR5 report \citep[ch.~13]{ipcc} combines the heat expansion of the ocean with temperature forced models for glacial melt, Greenland ice melt, and Antarctic ice melt and with land rise due to rebound from the last ice age. Judging from the supplementary material to \citet[ch.~13]{ipcc}, the uncertainty assessment is only based on the spread of the ensemble of temperature projections, not on the additional uncertainty in the ice models used.

We will instead use the empirical approach of Rahmstorf and collaborators \citep{Rahmstorf07,Rahmstorf11}, employing the statistical modeling of \citet{Bolin2014a} to relate global annual mean temperature anomalies to global mean sea level anomalies. 


We then apply the estimated historical relationship to projected temperatures from the CMIP5 experiment \citep{cmip5} to obtain projected global annual mean sea level, taking into account the uncertainty in the statistical model as well as the spread of the temperature projection ensemble (see subsection \ref{unc_ass}). 
For the i'th temperature projection $T_t^i$ we estimate the corresponding global mean sea level as
\begin{linenomath*}
\[H_t^{gl,i} = \int\limits_{{t_0}}^t {{\hat a} (T_u^i - {{\hat T}_0}} )du + {\varsigma _t},\]
\end{linenomath*}
where ${\hat a}$ and ${\hat T}_0$ are regression parameters of observed global sea level  rise on observed global temperature and $\varsigma_t$ the integrated time series regression error.


\subsection{Local sea level}
In order to get from global sea level projections to local ones, it is important to note that sea level rise is not uniform over the globe. Glacial and land ice melting affect the local sea level differently depending on where the melted ice is located.
%Gravitational effects
Another major effect in Fennoskandia is the land rise due to isostatic rebound from the glaciers of the last ice age. 
Again, we will use historical data to relate global sea level to isostatically corrected local sea level using a time series regression model. 
The local sea level projections are then obtained by first relating projected temperature to global sea level, and then relating the global sea level to the local one. Each climate model temperature projection yields a different local sea level projection. The local sea level projection based on the i'th climate model for years beyond 2000 is estimated as
\begin{linenomath*}
\[H_t^{loc,i} + \gamma (t -2000 ) = {\hat b} H_t^{gl,i}  + {\varepsilon _t},\]
\end{linenomath*}
where $\gamma$ is the annual land rise rate, $t$ denotes year, and $ {\hat b} $ is the regression coefficient relating global to local sea level.

\subsection{Uncertainty assessment}
\label{unc_ass}
Following the approach of \citet{Guttorp2014} we assess the uncertainty in the local sea level projections taking into account the variability between the climate projections used, the uncertainties in the regressions of global mean temperature on global mean sea level and of global on local sea level. We express the sea level projection uncertainty in terms of a confidence band that is simultaneously of the intended  level  for all projection years. This allows us, for example, to get a confidence band for the years when a given sea level rise is obtained. 

\subsection{Limitations of the sea level projections}
The main assumption is using historical relationships in statistical projections of the type used in this paper is that there is no major change in how temperature relates to sea level, globally and locally. Among the factors that may invalidate this approach are changes in water storage on land (in essence removing water from the oceans), excessive siphoning of groundwater (resulting in land subsidence), changes in the rates of glacial and land ice melt, and changes in Earth's gravitational field due to transfer of mass from land ice to ocean water. For example, the rate of ice melt on Greenland may suddenly increase substantially due to intense warming of both air and sea water \citep{bamber2013}. A recent paper \citep{jevrejeva2016} indicates that the upper tails of sea level rise may be substantially higher when taking into account expert assessment of land ice melting. Our current climate models are not able to resolve the ice processes sufficiently to include such so called tipping points into the projections. Also, the IPCC scenarios \citep{change} do not include changes in water usage (cf. \citet{wada2012}). 

\section{Timing of adaptation measures}
\label{decision_tools_PartI}

To assess the optimal timing of an adaptation measure, we employ a probabilistic extension of the framework described by \citep{Fankhauser&1999} in which we obtain a probabilistic distribution for the net present value damage in a given year for various adaptation options. The probabilistic distribution is constructed by considering uncertainty
\begin{itemize}
\item in the local sea level projections;
\item in the annual damage costs;
\item in the effect of changes in sea level on the annual damage costs. 
\end{itemize}

\subsection{Annual damage costs} 

We model the distribution of annual damage, $F_{\textup{d}, t_0}$, for the year $t_0 = 2015$ by the three parameter Burr distribution \citep{Burr1942} with density
\begin{linenomath*}
  \begin{equation}\label{eq:Burr}
  f_{\textup{d}, t_0} (x) = \frac{\alpha \gamma (x/ \theta)^{\gamma}}{x [ 1 + (x/ \theta)^{\gamma}]^{\alpha + 1}}
  \end{equation}
\end{linenomath*}
for $x > 0$, where $\alpha$ and $\gamma$ are shape parameters with $\alpha, \gamma > 0$, and $\theta >0$ is a scale parameter. The Burr distribution has a heavy upper tail and is commonly used to model damage loss, see e.g. \cite{Klugman&2012}. The parameters of the distribution are estimated using historical data for annual storm surge damage. Data prior to 2015 are adjusted to the 2015 level using the consumer price index. After adjustment, we assume stationarity over the period and independence between years.

Under a constant sea level, we can obtain a sample trajectory $\{d_{t_1}, d_{t_2}, \ldots, d_{t_{85}}\}$ of future annual damages for $t_1 = 2016, \ldots, t_{85} = 2100$ by drawing $85$ i.i.d. values from the estimated distribution $\hat{F}_{\textup{d}, t_0}$.  By repeating this process $J$ times, we obtain an empirical damage distribution for each future year $t_i$ given by
\begin{linenomath*}
  \[
  \hat{F}_{\textup{d}, t_i} (x) = \frac{1}{J} \sum_{j=1}^J \mathbbm{1}\Bigg\{ \frac{d_{t_i}^{(j)}}{\prod_{l \leq i} (1 + r_{t_l}) } \leq x \Bigg\}
  \]
\end{linenomath*}
for $i = 1, \ldots, 85$, where $r_{t_i}$ is the discount rate for year $t_i$. Alternatively, we obtain an empirical distribution of the total damage over the period $2016-2100$ by considering
\begin{linenomath*}
  \[
  d_{\textup{total}}^{(j)} = \sum_{i=1}^{85} \frac{d_{t_i}^{(j)}}{\prod_{l \leq i} (1 + r_{t_l})}
  \]
\end{linenomath*}
and similarly for the cumulative damage.  

\subsection{Effect of changes in sea level}

We assume that changes in sea level have a multiplicative effect on the annual damage cost. That is, for a sea level anomaly $s_{t_i}$ in year $t_i > 2015$ compared to the 2015 level, the annual damage cost becomes
\begin{linenomath*}
  \[
  g(s_{t_i} | \boldsymbol \beta ) d_{t_i},
  \]
\end{linenomath*}
where $g( \cdot | \boldsymbol \beta )$ is a monotonic positive function with parameter vector  $\boldsymbol \beta$ such that $g(s|\boldsymbol \beta ) > 1$ for $s>0$ and $g(s|\boldsymbol \beta ) < 1$ for $s < 0$. \cite{Hallegatte&2013} estimate a similar effect function valid in 2050 for $s \in \{0,20,40\}$ for 136 coastal cities. Here, we use their results for 15 European cites: Amsterdam, Athens, Barcelona, Dublin, Glasgow, Hamburg, Helsinki, Copenhagen, Lisbon, London, Marseilles, Naples, Porto, Rotterdam and Stockholm.  To obtain a city-specific effect function for a large range of sea level anomalies we employ a linear extrapolation as shown in Figure~\ref{fig:EffectFct}. We then obtain a sample of effect functions $\{g(\cdot|\boldsymbol \beta^{(j)})\}_{j=1}^J$ by sampling with replacement from this ensemble of trajectories with all 15 ensemble members considered equally probable. 

\begin{figure}[!hbpt]
\begin{center}
\includegraphics[width=0.5\linewidth]{DecisionAnalysisBergenPartII.pdf}
\caption{Relative change in mean annual damage as a function of sea level rise for 15 European cites as estimated by \cite{Hallegatte&2013} (black circles) with linearly extrapolated values indicated by gray lines. The median change and the corresponding extrapolation are indicated in red.}
\label{fig:EffectFct}
\end{center}
\end{figure}


Let further $\{s_{t_1}^{(j)}, \ldots, s_{t_{85}}^{(j)} \}_{j=1}^J$ denote a sample of projections for annual sea level anomalies compared to the 2015 value. An empirical damage distribution for the future year $t_i$ that accounts for uncertainty in damage, sea level rise and its effect on the damage is then given by
\begin{linenomath*}
  \begin{equation}\label{eq:FutureDamage}
    \hat{F}_{\textup{d}, t_i}^\textup{s} (x) = \frac{1}{J} \sum_{j=1}^J \mathbbm{1}\Bigg\{ \frac{g(s_{t_i}^{(j)}| \boldsymbol \beta^{(j)}) d_{t_i}^{(j)}}{\prod_{l \leq i} (1 + r_{t_l})} \leq x \Bigg\}.
  \end{equation}
\end{linenomath*}

The distribution in \eqref{eq:FutureDamage} describes the projected damage distribution with no adaptation measures. In addition, we can incorporate an adaptation measure of cost $C$ that protects against $K$ cm of increased sea level from year $t_k$ onward. This results in a damage distribution given by
\begin{linenomath*}
  \[
  \hat{F}_{\textup{d}, t_i}^{\textup{s}, \textup{a}_k} (x) =
    \begin{cases}
      \frac{1}{J} \sum_{j=1}^J \mathbbm{1}\Bigg\{ \frac{g(s_{t_i}^{(j)}| \boldsymbol \beta^{(j)}) d_{t_i}^{(j)}}{\prod_{l \leq i} (1 + r_{t_l})} \leq x \Bigg\}, & t_i < t_k \\[2ex]
      \frac{1}{J} \sum_{j=1}^J \mathbbm{1}\Bigg\{ \frac{g(s_{t_i}^{(j)} - K| \boldsymbol \beta^{(j)}) d_{t_i}^{(j)} + C}{\prod_{l \leq i} (1 + r_{t_l})} \leq x \Bigg\}, & t_i = t_k \\[2ex]
      \frac{1}{J} \sum_{j=1}^J \mathbbm{1}\Bigg\{ \frac{g(s_{t_i}^{(j)} - K| \boldsymbol \beta^{(j)}) d_{t_i}^{(j)}}{\prod_{l \leq i} (1 + r_{t_l})} \leq x \Bigg\}, & t_i > t_k. \\
    \end{cases}
  \]
\end{linenomath*}

\subsection{Limitations of the decision framework}
The main limitation of this light touch decision framework is that we have significantly simplified the assessment of the effect of sea level rise on the damage costs. In particular, the linear extrapolation of the results reported in \citet{Hallegatte&2013} might provide a conservative estimate of the effect of extreme sea level rise. However, with only two data points, extrapolation approaces such as a power law or exponential growth seem unfeasible. Alternatively, a modeling framework similar to that of \citet{Hallegatte&2013} could be applied directly to a larger range of potential changes in sea level. Additionally, we have not specifically accounted for potential changes in storm surge patterns.

\section{Danish decision framework}

\section{Case studies}
\label{cases}

\subsection{Data}
The historical global mean temperature series is obtained from \citet{giss}. Climate projections of global mean temperature are from the fifth climate model intercomparison project, CMIP5 \citep{cmip5}. The global mean sea level series is obtained from \citet{csiro}. We use local tide gauge data from the Permanent Service for Mean Sea Level, UK, which is the worldwide repository for national sea level data. Glacial isostatic adjustment for Bergen is obtained from \citet{Simpson2014}, and for Esbjerg in personal communication from Peter Thejll at the Danish Meteorological Institute. 

The Bergen monthly series is missing data for 62 months, including all of the years 1942--43. To deal with occasional short stretches of missing data (at most one or two months) we use median polish replacement \citep{medpol} and then compute annual averages. For the years 1942-43, we use use the average difference between Bergen and the average of all other Norwegian stations in 1940 and 1943 to estimate values for 1941 and 1942, using the average of all other Norwegian stations corrected by the average difference. 

The Esbjerg monthly series is missing data for 19 months. Here, too, we use median polish to fill in missing data and then compute annual averages.

Annual damage costs for the Bergen case study are obtained from the Norwegian Natural Perils Pool (NPP;  data are available at \url{https://www.finansnorge.no/statistikk/skadeforsikring/Naturskadestatistikk-NASK/}). The NPP data are available for the period 1980-2015 and are aggregated to a county level. For improved parameter estimation, we include the data from Rogaland county which is the county directly south of Hordaland and with similar characteristics. We use a dicsount rate of 4\% for the first 40 years of the analysis, a rate of 3\% for 40 to 75 years into the future and a rate of 2\% beyond 75 years (cf. Section 5.8 of \citet{DiscountRate}).  

\subsection{Sea level rise in Bergen and Esbjerg}

Figure \ref{fig:bergenobs} shows uncorrected and corrected Bergen sea level data, and the relationship between the corrected Bergen data and the global sea level data. The glacial isostatic adjustment is 0.26 (0.07) cm/yr. The time series regression uses an ARMA(1,1)-model \citep{boxjenkins}, with AR parameter 0.82 (0.13), and MA parameter --0.61 (0.17). The regression slope is 1.30 (0.12).

\begin{figure}[!hbpt]
\begin{center}
\includegraphics[width=0.75\linewidth]{bergenfit_edit.png}
\caption{ The left figure shows raw (black) and gia-corrected (red) sea level data from Bergen, The right figure relates the gia-corrected Bergen sea level to the global sea level series of \citet{csiro}. The straight line is the time series regression line.}
\label{fig:bergenobs}
\end{center}
\end{figure}

For the relationship between global annual mean temperature and global annual mean sea level rise we use the results from \citet{Bolin2014a}. The left panel of figure \ref{fig:ci} shows the simultaneous 90 \% confidence region for Bergen sea level rise relative to 1999 under scenario RCP 8.5, which is the scenario Norwegian authorities recommend for planning purposes.
\begin{figure}[!hbpt]
\begin{center}
\begin{minipage}{.5\textwidth}

 \includegraphics[width=\linewidth]{bergen_ci.pdf}
 % \label{fig:bergenci}

\end{minipage}%
\begin{minipage}{.5\textwidth}

    \includegraphics[width=\linewidth]{esbjerg_ci.pdf}
 % \label{fig:esbjergci}

\end{minipage}
\caption{Simultaneous 90\% confidence set (thick black lines) for Bergen (left) and Esbjerg( right) sea level projections for the years 2000-2100 using RCP8.5. The sea level data are shown in blue and end in 2015. The thin red lines are the projections without uncertainty based on each of the climate models. The dashed purple lines connect pointwise confidence intervals for each year. }
\label{fig:ci}
\end{center}
\end{figure}

For Esbjerg, the glacial isostatic adjustment is 0.06 (0.03) cm/yr. The time series regression model relating gia-corrected local to global sea level is an MA(1) model with parameter 0.17 (0.09). The regression slope is 1.02 (0.06). The right panel of figure \ref{fig:ci} shows the simultaneous 90\% confidence region for sea level rise relative to 1999 under scenario RCP 8.5.


\subsection{Timing of adaptation measures {\color{blue} (KdB, TT)}}

The histogram of the observed annual damages and the estimated distribution are given in Figure~\ref{fig:BergenDamageDist}. The parameter estimates for the Burr distribution are $\hat{\alpha} = 1.27$, $\hat{\gamma} = 0.51$ and $\hat{\theta} = 0.002$. 

\begin{figure}
\begin{center}
\includegraphics[width=\linewidth]{DecisionAnalysisBergen.pdf}
\caption{Damage data used in the decision analysis for Bergen: (a) Estimated distribution of annual storm surge damage in Bergen for 2015 (red) based on observed annual damage in Hordaland and Rogaland 1980-2015 (gray bars); (b) Relative change in mean annual damage as a function of sea level rise for 15 European cites as estimated by \cite{Hallegatte&2013} (black circles) with linearly extrapolated values indicated by gray lines. The median change and the corresponding extrapolation are indicated in red.}
\label{fig:BergenDamageDist}
\end{center}
\end{figure}




\begin{figure}[!hbpt]
\begin{center}
\includegraphics[width=0.5\linewidth]{CumDamageCostsBergen.pdf}
\caption{Distribution of accumulated future loss 2016-2100 under RCP 8.5 and assuming that no adaptation measures are implemented.} 
\label{fig:NoAction}
\end{center}
\end{figure}


\subsection{Selection of adaptation measures(?) {\color{blue} (MD)}}

A case study focusing on Denmark. 

\section{The value of including uncertainty}
\label{unc}

In many cases sea level rise projections are given as a single number for each scenario, usually the mean or median of the ensemble of projections from different climate models (e.g. \citet{climateimpactgroup}). Sometimes the spread of the ensemble is used to assess the uncertainty in the projections (e.g., the Norwegian Environmental Agency recommends using the upper ensemble value for RCP 8.5 as the basis for planning decision, pers. comm. from Even Nilsson, Norwegian Mapping Authority). In our analysis there are two more sources of uncertainty, namely the two regression models. Figure \ref{fig:unc} shows the single number (vertical black line), the ensemble spread (histogram), the uncertainty including only the global model (red) and the full uncertainty (blue) for Bergen projections of sea level rise relative to 1999 under RCP 8.5. We see that the ensemble range is about 16 cm, whereas the overall uncertainty range is about 40 cm.


\begin{figure}[!hbpt]
\begin{center}
\includegraphics[width=0.5\linewidth]{unc.pdf}
\caption{2050 Bergen sea level projections with uncertainty due to different sources for RCP 8.5. The black vertical line is the median projection (with no uncertainty), while the gray histogram corresponds to the spread of the climate models, the red curve adds the uncertainty due to the relation between global temperature and global sea level, and the blue line that due to downscaling global sea level to Bergen. } 
\label{fig:unc}
\end{center}
\end{figure}





%  ACKNOWLEDGMENTS
\begin{acknowledgments}
This work was funded by NordForsk through project number 74456 ``Statistical Analysis of Climate Projections'' (eSACP) and The Research Council of Norway through project number 243953 ``Physical and Statistical Analysis of Climate Extremes in Large Datasets'' (ClimateXL). The source code for the analysis is implemented in the statistical programming language {\tt R} (\citet{R}) and is available on GitHub at \url{http://github.com/eSACP/SeaLevelDecisions/Code}.
\end{acknowledgments}

%%  REFERENCE LIST AND TEXT CITATIONS
% 5\bibliographystyle{../BibTeX/agufull08}

\bibliography{ref.bib}
% Please use ONLY \citet and \citep for reference citations.

%\begin{thebibliography}{37}
%%   Before submitting: copy all the contents into the .bbl LaTeX file here
%%   and run latex again
%\providecommand{\natexlab}[1]{#1}
%\expandafter\ifx\csname urlstyle\endcsname\relax
%  \providecommand{\doi}[1]{doi:\discretionary{}{}{}#1}\else
%  \providecommand{\doi}{doi:\discretionary{}{}{}\begingroup
%  \urlstyle{rm}\Url}\fi


%\end{thebibliography}


%% Enter Figures and Tables here:

\end{document}
