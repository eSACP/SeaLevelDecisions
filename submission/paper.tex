%\documentclass[wrr]{agutex}
\documentclass[wrr, draft]{agutex}
% Author names in capital letters:
\authorrunninghead{1st AUTHOR ET AL.}

% Shorter version of title entered in capital letters:
\titlerunninghead{ADAPTING TO UNCERTAIN SEA LEVEL RISE}

%Corresponding author mailing address and e-mail address:
\authoraddr{Corresponding author: Name, Address. (email)}

\usepackage{bbm, amsmath}
%\usepackage[dvips]{graphicx}
\usepackage{graphicx}
\setkeys{Gin}{draft=false}
\usepackage{paralist, booktabs}
\usepackage{url}
\usepackage{color}


\usepackage{lineno}
 \linenumbers*[1]
%  To add line numbers to lines with equations:

\begin{document}

%% ------------------------------------------------------------------------ %%
%  TITLE
%% ------------------------------------------------------------------------ %%

\title{I don't know, are you sure you want to do this?}

%% ------------------------------------------------------------------------ %%
%  AUTHORS AND AFFILIATIONS
%% ------------------------------------------------------------------------ %%

% \altaffilmark will produce footnote;
% matching \altaffiltext will appear at bottom of page.

\authors{T. Thorarinsdottir.\altaffilmark{1}, P. Guttorp.\altaffilmark{1},...}

\altaffiltext{1}{Norwegian Computing Centere}

\altaffiltext{2}{Danish Technological University}

\altaffiltext{3}{Affiliation three}

%% ------------------------------------------------------------------------ %%
%  ABSTRACT
%% ------------------------------------------------------------------------ %%

\begin{abstract}
...
\end{abstract}


%% ------------------------------------------------------------------------ %%
%  BEGIN ARTICLE
%% ------------------------------------------------------------------------ %%

\begin{article}



\section{Introduction}\label{sec:intro}

\section{Sea level projections {\color{blue} (PG)}}



\subsection{Global sea level}

\subsection{Local sea level}

\subsection{Uncertainty assessment}


\subsection{Limitations of the sea level projections}

\section{Decision tools {\color{blue} (KdB, MD, TT)}}

\subsection{Timing of adaptation measures}

We consider adaptation decision making related to the timing of proactive adaptation measures. That is, the goal is to adapt to sea level rise before major damages occur. In a cost-benefit framework, an investment should be delayed as long as the benefits of delay (avoided investment costs) are greater than the associated costs (higher climate change damages) \citep{Fankhauser&1999}.

\cite{Fankhauser&1999} describe a deterministic framework where an adaptation investment of $C^0$ now (at time $n=0$) leads to unmitigated damage of $d_0^0$ in period $0$, and a stream of partially mitigated damages $d_t^0$ in periods $t=1,2,\ldots$. If $r$ is the discount rate, the net present value damage, $D^0$, associated with this investment is
\begin{linenomath*}
\begin{equation}\label{eq:deterministic damage}
D^0 = C^0 + d_0^0 + \frac{d_1^0}{1+r} + \frac{d_2^0}{(1+r)^2} + \cdots  
\end{equation}
\end{linenomath*}
In comparison, postponing the adaptation investment to time period $n=1$ would lead to unmitigated damages in periods $0$ and $1$, and partially mitigated damages, $d_t^1$, thereafter. The delay would be preferable if
\begin{linenomath*}
\[
C^0 - \frac{C^1}{(1+r)} > (d_0^1 - d_0^0) + \frac{d_1^1 - d_1^0}{1+r} + \frac{d_2^1 - d_2^0}{(1+r)^2} + \cdots
\]
\end{linenomath*}
Here, the expression on the left describes the benefits of the delay while the expression on the right describes the cost of the delay. In the simplest case, there is no change in investment costs ($C^0 = C^1 = C$) and the delay has no lasting effects beyond period 1 ($d_t^1 = d_t^0$ for $t > 1$). In this case, the comparison is between the expected return $r$ earned on the captial while implementation is delayed and one addtional time period of unmitigated damage,
\begin{linenomath*}
  \[
  r C > d_1^1 - d_1^0.
  \]
  \end{linenomath*}

\subsection{Limitations of the decision framework}

\section{Case studies}

\subsection{Data {\color{blue} (PG)}}

\subsection{Timing of adaptation measures {\color{blue} (KdB, TT)}}

A case study focusing on and comparing different cities in Norway.

\subsection{Selection of adaptation measures(?) {\color{blue} (MD)}}

A case study focusing on Denmark. 

\section{Conclusions}

%  ACKNOWLEDGMENTS
\begin{acknowledgments}
This work was funded by NordForsk through project number 74456 ``Statistical Analysis of Climate Projections'' (eSACP) and The Research Council of Norway through project number 243953 ``Physical and Statistical Analysis of Climate Extremes in Large Datasets'' (ClimateXL). The source code for the analysis is implemented in the statistical programming language {\tt R} (\url{http://www.R-project.org}) and is available on GitHub at \url{http://github.com/eSACP/...}.
\end{acknowledgments}

%%  REFERENCE LIST AND TEXT CITATIONS
% 5\bibliographystyle{../BibTeX/agufull08}
\bibliographystyle{agufull08}
% \bibliography{ref.bib}
% Please use ONLY \citet and \citep for reference citations.

\begin{thebibliography}{37}
%%   Before submitting: copy all the contents into the .bbl LaTeX file here
%%   and run latex again
\providecommand{\natexlab}[1]{#1}
\expandafter\ifx\csname urlstyle\endcsname\relax
  \providecommand{\doi}[1]{doi:\discretionary{}{}{}#1}\else
  \providecommand{\doi}{doi:\discretionary{}{}{}\begingroup
  \urlstyle{rm}\Url}\fi

\bibitem[{\textit{Fankhauser et~al.}(1999)}]{Fankhauser&1999}
  Fankhauser, S., J.~B.~Smith, and R.~S.~J.~Tol (1999), {Weathering climate
    change: some simple rules to guide adaptation decisions},
  \textit{Ecological Economics}, \textit{30}, 67--78.
  
\bibitem[{\textit{Robert and Casella}(2004)}]{RobertCasella2004}
Robert, C.~P., and G.~Casella (2004), \textit{Monte Carlo Statistical Methods},
  2nd ed., Springer, New York.

\bibitem[{\textit{Rue et~al.}(2009)\textit{Rue, Martino, and
  Chopin}}]{Rueetal2009}
Rue, H., S.~Martino, and N.~Chopin (2009), {Approximate {B}ayesian Inference
  for Latent {G}aussian Models Using Integrated Nested {L}aplace Approximations
  (with Discussion)}, \textit{Journal of the Royal Statistical Society, Series
  B}, \textit{71}, 319--392.


\end{thebibliography}

%% ------------------------------------------------------------------------ %%
%  END ARTICLE
%% ------------------------------------------------------------------------ %%
\end{article}

%% Enter Figures and Tables here:

\end{document}
