%\documentclass[wrr]{agutex}
\documentclass[wrr, draft]{agutex}
% Author names in capital letters:
\authorrunninghead{1st AUTHOR ET AL.}

% Shorter version of title entered in capital letters:
\titlerunninghead{ADAPTING TO UNCERTAIN SEA LEVEL RISE}

%Corresponding author mailing address and e-mail address:
\authoraddr{Corresponding author: Name, Address. (email)}

\usepackage{bbm, amsmath}
%\usepackage[dvips]{graphicx}
\usepackage{graphicx}
\setkeys{Gin}{draft=false}
\usepackage{paralist, booktabs}
\usepackage{url}
\usepackage{color}


\usepackage{lineno}
 \linenumbers*[1]
%  To add line numbers to lines with equations:

\begin{document}

%% ------------------------------------------------------------------------ %%
%  TITLE
%% ------------------------------------------------------------------------ %%

\title{I don't know, are you sure you want to do this?}

%% ------------------------------------------------------------------------ %%
%  AUTHORS AND AFFILIATIONS
%% ------------------------------------------------------------------------ %%

% \altaffilmark will produce footnote;
% matching \altaffiltext will appear at bottom of page.

\authors{T. Thorarinsdottir\altaffilmark{1}, P. Guttorp\altaffilmark{1}, M. Drews\altaffilmark{2}}

\altaffiltext{1}{Norwegian Computing Centere}

\altaffiltext{2}{Danish Technological University}

\altaffiltext{3}{Affiliation three}

%% ------------------------------------------------------------------------ %%
%  ABSTRACT
%% ------------------------------------------------------------------------ %%

\begin{abstract}
...
\end{abstract}


%% ------------------------------------------------------------------------ %%
%  BEGIN ARTICLE
%% ------------------------------------------------------------------------ %%

\begin{article}



\section{Introduction}\label{sec:intro}

\section{Sea level projections {\color{blue} (PG)}}



\subsection{Global sea level}
Most climate models do not explicitly provide sea level as an output of the calculations. 

We will use the empirical approach of Rahmstorf and collaborators \citep{Rahmstorf07,Rahmstorf11}, employing the statistical modeling of \citet{Bolin2014a} to relate global annual mean temperature anomalies \citep{giss} to global mean sea level anomalies \citep{csiro}. 

%Figure showing historical relationship

We then apply the estimated historical relationship to projected temperatures from the CMIP5 experiment \citep{cmip5} to obtain projected global annual mean sea level. 
%Figure with sea level projections & IPCC projections


\subsection{Local sea level}
In order to get from global sea level projections to local ones, it is important to note that sea level rise not is uniform over the globe. Glacial and land ice melting affect the local sea level differently depending on where the melted ice is located.
%Gravitaional effects
Another major effect in Fennoskandia is the land rise due to isostatic rebound from the glaciers of the last ice age. 
Again, we will use historical data to relate global sea level to isostatically corrected local sea level using a time series regression model. 
%Figure showing corrected and uncorrected Bergen data and regression model.

The local sea level projections are then obtained by first relating projected temperature to global sea level, and then relating the global sea level to the local one. Each climate model temperature projection yields a different local sea level projection.

\subsection{Uncertainty assessment}
Following the approach of \citet{Guttorp2014} we assess the uncertainty in the local sea level projections taking into account the variability between the climate projections used, the uncertainties in the regressuions of global mean temperature on global mean sealevel and of global on local sea level. We express the sea level projection uncertainty in terms of a confidence band that is simultaneously of the intended  level  for all projection years. This allows us, for example, to get a confidence band for the years when a given sea level rise is obtained.

%Figure showing Bergen and Copenhagen slr projections

\subsection{Limitations of the sea level projections}
The main assumption is using historical relationships in statistical projections of the type used in this paper is that there is no major change in how temperature relates to sea level, globally and locally. Among the factors that may invalidate this approach are changes in water storage on land, changes in the rates of glacial and land ice melt, and changes in Earth's gravitational field due to transfer of mass from land ice to ocean water. For example, the rate of ice melt on Greenland may suddenly increase substantially due to intense warming of both air and sea water. Our current climate models are not able to resolve the ice processes sufficiently to include such so called tipping points into the projections.

\section{Decision tools {\color{blue} (KdB, MD, TT)}}

\subsection{Timing of adaptation measures}

We consider adaptation decision making related to the timing of proactive adaptation measures. That is, the goal is to adapt to sea level rise before major damages occur. In a cost-benefit framework, an investment should be delayed as long as the benefits of delay (avoided investment costs) are greater than the associated costs (higher climate change damages) \citep{Fankhauser&1999}.

\cite{Fankhauser&1999} describe a deterministic framework where an adaptation investment of $C^0$ now (at time $n=0$) leads to unmitigated damage of $d_0^0$ in period $0$, and a stream of partially mitigated damages $d_t^0$ in periods $t=1,2,\ldots$. If $r$ is the discount rate, the net present value damage, $D^0$, associated with this investment is
\begin{linenomath*}
\begin{equation}\label{eq:deterministic damage}
D^0 = C^0 + d_0^0 + \frac{d_1^0}{1+r} + \frac{d_2^0}{(1+r)^2} + \cdots  
\end{equation}
\end{linenomath*}
In comparison, postponing the adaptation investment to time period $n=1$ would lead to unmitigated damages in periods $0$ and $1$, and partially mitigated damages, $d_t^1$, thereafter. The delay would be preferable if
\begin{linenomath*}
\[
C^0 - \frac{C^1}{(1+r)} > (d_0^1 - d_0^0) + \frac{d_1^1 - d_1^0}{1+r} + \frac{d_2^1 - d_2^0}{(1+r)^2} + \cdots
\]
\end{linenomath*}
Here, the expression on the left describes the benefits of the delay while the expression on the right describes the cost of the delay. In the simplest case, there is no change in investment costs ($C^0 = C^1 = C$) and the delay has no lasting effects beyond period 1 ($d_t^1 = d_t^0$ for $t > 1$). In this case, the comparison is between the expected return $r$ earned on the captial while implementation is delayed and one addtional time period of unmitigated damage,
\begin{linenomath*}
  \[
  r C > d_1^1 - d_1^0.
  \]
  \end{linenomath*}

\subsection{Limitations of the decision framework}

\section{Case studies}

\subsection{Data {\color{blue} (PG)}}

\subsection{Timing of adaptation measures {\color{blue} (KdB, TT)}}

A case study focusing on and comparing different cities in Norway.

\subsection{Selection of adaptation measures(?) {\color{blue} (MD)}}

A case study focusing on Denmark. 

\section{Conclusions}

%  ACKNOWLEDGMENTS
\begin{acknowledgments}
This work was funded by NordForsk through project number 74456 ``Statistical Analysis of Climate Projections'' (eSACP) and The Research Council of Norway through project number 243953 ``Physical and Statistical Analysis of Climate Extremes in Large Datasets'' (ClimateXL). The source code for the analysis is implemented in the statistical programming language {\tt R} (\url{http://www.R-project.org}) and is available on GitHub at \url{http://github.com/eSACP/...}.
\end{acknowledgments}

%%  REFERENCE LIST AND TEXT CITATIONS
% 5\bibliographystyle{../BibTeX/agufull08}
\bibliographystyle{agufull08}
\bibliography{ref.bib}
% Please use ONLY \citet and \citep for reference citations.

%\begin{thebibliography}{37}
%%   Before submitting: copy all the contents into the .bbl LaTeX file here
%%   and run latex again
%\providecommand{\natexlab}[1]{#1}
%\expandafter\ifx\csname urlstyle\endcsname\relax
%  \providecommand{\doi}[1]{doi:\discretionary{}{}{}#1}\else
%  \providecommand{\doi}{doi:\discretionary{}{}{}\begingroup
%  \urlstyle{rm}\Url}\fi


%\end{thebibliography}

%% ------------------------------------------------------------------------ %%
%  END ARTICLE
%% ------------------------------------------------------------------------ %%
\end{article}

%% Enter Figures and Tables here:

\end{document}
