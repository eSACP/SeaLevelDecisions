% $Id: manual.tex 239 2013-04-18 13:25:02Z wolfgang $
% Specify language of the report

\documentclass[note,screen,british,11pt]{nrdoc}


% Verbatim is already loaded by nrdoc. It is only
% included to make LaTeX2html work better.
\usepackage{verbatim}   

% -------------------------------------------------
% BEGIN nrdoc special commands
\ifx\nrdocument\undefined\relax\else
\reportnumber{sand/08/05}
%\publicationyear{2003}
\keywords{\LaTeX, NR}
\projectnumber{}
\project{}
\availability{Internal}
\frontpagefigure{figs/texfriendly} % No file extension.
\target{All employees}
\researchfield{}
\fi
% END nrdoc special commands
% ------------------------------------------------


\makeindex

\begin{document}

\title{User manual for nrdoc: a \LaTeX\ class for notes and reports}
\author{Harald H. Soleng \email{harald.soleng@nr.no} \and 
         Anders L�land \email{anders.loland@nr.no}}
\aboutauthors{The authors are local \LaTeX\ gurus}
\date{\today}
\maketitle

% The abstract must always be present
\begin{abstract}
This document gives an example of the use of the \texttt{nrdoc} class. The
purpose of this document class is to provide a unified style for NR
notes and reports conforming with standard typographical practice in
Norwegian and English.  
The \texttt{nrdoc} class was originally
written by Harald H. Soleng and Anders L�land,  
redesigned by Lars Brusletto Sveen of Munch
Design, and again modified by Harald H. Soleng and Anders L�land.
\end{abstract}

\tableofcontents

\section{Introduction}
The \texttt{nrdoc} class was first written by \cite{NRDocSource03,NRDocManual03}
Harald H. Soleng and Anders L�land. When   Munch
Design redesigned the graphical profile of the Institute, the class was
modified
by Lars Brusletto Sveen. Later, 
the original authors have fine tuned the class 
and added more functionality.

This document gives an example of the use of the \texttt{nrdoc} class. 
In Appendix \ref{sec:code} the \LaTeX\ source code for this document is
given.  

\section{Two ways to use the nrdoc class}

The \texttt{nrdoc} class can be used both by \LaTeX\ \index{latex@\LaTeX} 
\citep{latex}
(using the \texttt{latex} command) and 
pdf\LaTeX\ \index{pdflatex@pdf\LaTeX}
(using the \texttt{pdflatex} command). Please note that you need to use 
a reasonably recent tex distribution. Using the option
\textit{screen} \index{screen}\index{option!screen} together with
\texttt{pdflatex} or \texttt{latex} and
\texttt{dvipdfm}\index{dvipdfm} gives an \textbf{electronic report
  with hyperlinks}.\index{hyperlinks} An example of the hyperlinks is
given here: If this is an electronic report, Equations \eqref{eq:ey}
can be found by clicking the equation number. The option
\textit{screen} gives coloured hyperlinks, so
this option should not be used for a paper document. 

By default the page layout is two-sided\footnote{On older \LaTeX\
installations, the default is one-sided. If you want to be sure to 
get two-sided output, specify the additional option
\textit{twoside}\index{option!twoside} in the \texttt{documentclass}
command. The option
\textit{oneside}\index{option!oneside} gives one-sided page layout.}.


\subsection{Note and report}
The default option \textit{note}\index{option!note} gives a NR
note. The option  \textit{report}\index{option!report} gives a NR
report. For a NR report, additional variables described in Section
\ref{sec:fronttitle} must be specified.



\subsection{Long reports}

The class option \textit{long}\index{option!long} 
gives access to part\index{part} and chapter\index{chapter} commands.
By default the highest level is section\index{section}.



\section{Front and title page\label{sec:fronttitle}}

For the title page to work, the abstract\index{abstract} (summary) must be
present. The title is set by \verb!\title!. \index{title}
\verb!\nrtitle! \index{nrtitle} can be used to specify the title in
the title page (this is convenient if the title on the front page
includes some line shifts). \verb!\shorttitle! \index{nrtitle}
specifies the text in the heading of each page. If \verb!\shorttitle!
\index{shorttitle} is not specified, the text in the heading will be
taken from \verb!\title!. 

The date\index{date} is taken from \verb!\date!.

The note or publication number is given by
\verb+\reportnumber+\index{reportnumber@\verb+\reportnumber+}.
The keywords of the document are given by
\verb+\keywords+\index{keywords@\verb+\keywords+}.
The project title is given by 
\verb+\project+\index{project@\verb+\project+},
the project number by
\verb+\projectnumber+\index{projectnumber@\verb+\projectnumber+},
 and the target group by
\verb+\target+\index{target@\verb+\target+}.
The research field is specified using
\verb+\researchfield+\index{researchfield@\verb+\researchfield+}. Optionally, the name of a quality assurerer may be specified by
\verb+\qa+\index{quality assurance@\verb+\qa+}.

For reports, you also need to supply the isbn\index{isbn number} number. 
The isbn number consists
of a country code, a publisher code, 
a publication number and a control digit.
You can either use \texttt{isbn}\index{isbn@\verb+\isbn+} to supply 
the full number or 
\texttt{shortisbn}\index{shortisbn@\verb+\shortisbn+}
to supply just the last five digits:
\begin{verbatim}
\isbn{82-539-0481-9}
\shortisbn{0481-9}
\end{verbatim}

The availability level is given by
\verb+\availability+\index{availability@\verb+\availability+}. There
are four possible levels of availability:
\begin{itemize}
\item Open\index{availability!open}: Open for everyone. Available
from NR's web site.
\item Internal\index{availability!internal}: Open for everyone at NR. 
\item Confidential\index{availability!confidential}: Confidential. 
\item Strictlyconfidential\index{availability!strictly confidential}:
Strictly confidential. 
\item Statoil restricted\index{availability!Statoil restricted}: Statoil restricted. 
\end{itemize}

A typical preamble\index{preamble} then looks as follows:
\begin{verbatim}

\documentclass[note,screen,twoside,british,12pt]{nrdoc}
\reportnumber{sand/03/03}
\keywords{\LaTeX, NR}
\project{Fun with \LaTeX}
\projectnumber{314259}
\frontpagefigure{figs/texfriendly} % No file extension.
\target{All employees}
\researchfield{}
\availability{Internal}
\end{verbatim}
At the beginning of the document, we specify the title,
author(s) and abstract:
\begin{verbatim}
\makeindex
\begin{document}
\title{User manual for nrdoc: a \LaTeX\ class for notes and reports}
\author{Harald Soleng \email{harald.soleng@nr.no} \and 
        Anders L�land \email{anders.loland@nr.no}}

\aboutauthors{The authors are local \LaTeX\ Gurus}
\date{\today}

\maketitle
\end{verbatim}
Separate each author with the \verb+\and+\index{and@\verb+\and+}
command. The email\index{email@\verb+\email+} address and the
\verb+\aboutauthors+\index{aboutauthors@\verb+\aboutauthors+} command are optional.
\begin{verbatim}
\begin{abstract}
The abstract is obligatory!
\end{abstract}

\tableofcontents        % optional
\end{verbatim}

\section{Including a figure}

On the front page, a figure\index{cover figure} can be included with
\begin{verbatim}
\frontpagefigure{figs/my_figure}
\end{verbatim}
\index{frontpagefigure@\verb+\frontpagefigure+}
Just give the full path to the figure file,
without the file extension. The figure is automatically scaled to fit
into the front page. A more low-level commands called
\texttt{frontpagefigure\-command}%
\index{frontpagefigurecommand@\verb+\frontpagefigurecommand+} 
allows more control. With this commands one 
has to specify a full
figure inclusion command, e.g., 
\begin{verbatim}
\frontpagefigurecommand{%
\includegraphics[width=50mm,height=40mm]{figs/my_figure}}
\end{verbatim}
If you want more than one figure on the cover 
you either need to combine them
using a graphics manipulation program, or 
use \texttt{frontpage\-figure\-command}
with several 
\texttt{include\-graphics}%
\index{includegraphics@\verb+\includegraphics+} 
commands.

Other figures\index{graphics}\index{illustration} \index{figure}
are included by \texttt{includegraphics}. An example is
given here (and in Figure \ref{fig:logo}):
\begin{verbatim}
\begin{figure}[ht]
  \begin{center}
    \includegraphics[width=0.3\linewidth,angle=45]{figs/fotball}
    \caption{\label{fig:logo} Rolling NR ball.}
  \end{center}
\end{figure}
\end{verbatim}
\begin{figure}[ht]
  \begin{center}
    \includegraphics[width=0.3\linewidth,angle=45]{figs/fotball}
    \caption{\label{fig:logo} Rolling NR ball.}
  \end{center}
\end{figure}
Note that the extension of the figure file name should be dropped. The
reason is that \LaTeX\ will find the required file itself. 
The figure
files should have the following format:\index{format}
\begin{description}
\item [\LaTeX] postscript [.ps] or encapsulated postscript
  [.eps] \index{format!postscript}
\item [pdf\LaTeX] pdf  [.pdf], jpeg [.JPG/.JPEG/.jpeg/.jpg] or png [.png] formats
  \index{format!pdf} \index{format!jpeg}\index{format!png}
\end{description}
We recommend pdf\LaTeX, since this gives instant pdf files and enables
inclusion of jpeg and png images (in addition to pdf files).

\subsection{Format conversion}
postscript files can be converted to pdf using the \texttt{epstopdf}
command or Acrobat Distiller. 
pdf files can be converted to postscript using the \texttt{pdftops}
command or Adobe Acrobat. 

\section{Fonts}

The default size is 11pt, but 10pt and 12pt sizes are also supported
through standard 
options.\index{option!10pt}\index{option!11pt}\index{option!12pt}

The standard font is type 1 versions of palatino. An important exception is
the typewriter font, which has been replaced with almost European fonts.


\section{Mathematics}\index{mathematics}

The \texttt{nrdoc} class loads the ams\index{amslatex@ams\LaTeX} 
packages so that the full power of these extensions are available.
\begin{align}\label{eq:ey}
\begin{split}
  EY(s)&=\sum_{i=1}^n \mathcal{K}_s(s_i)Ew(s_i),\\
  \gamma(s,s')&=\sum_{i=1}^n \mathcal{K}_s(s_i)
  \mathcal{K}_{s'}(s_i)Ew^2(s_i).
\end{split}
\end{align}

\section{The example environment\index{example}}

The example environment enables inclusion of numbered examples in your
notes and reports.
\begin{example}\caption{Example of example --  a quote!}
\begin{quote}
"Dirty-looking rascals, but I suppose every one has some little
immortal spark concealed about him. You would not think it, to look at
them. There is no a priori probability about it. A strange enigma is
man!" 

"Winwood Reade is good upon the subject," said Holmes. "He
remarks that, while the individual man is an insoluble puzzle, in the
aggregate he becomes a mathematical certainty. You can, for example,
never foretell what any one man will do, but you can say with
precision what an average number will be up to. Individuals vary, but
percentages remain constant. So says the statistician." 

\textit{--The Sign of Four, Arthur Conan Doyle}
\end{quote}
\end{example}
Note that each example must have a title, which is given in a caption;
\begin{verbatim}
\begin{example}\caption{Example of example --  a quote!}
\begin{quote}
"Dirty-looking rascals, but I suppose every one has some little
...
percentages remain constant. So says the statistician." 

\textit{--The Sign of Four , Arthur Conan Doyle}
\end{quote}
\end{example}
\end{verbatim}
A list of examples can be generated using the command \verb+\listofexamples+\index{listofexamples@\verb+\listofexamples+}.

\section{Citation using {Bib}\TeX}
 
Citations\index{cite}
are handled by {Bib}\TeX\ 
\index{bibtex@BiB\TeX}\index{cite!BiB\TeX@bibtex} and 
\texttt{natbib} \citep{natbib:7.1}. By default 
an author--year citation 
scheme\footnote{With the options \texttt{citealphanumeric}\index{cite!numeric citation label}\index{option!citealphanumeric}
or \texttt{citenumeric}\index{option!citenumeric}
a scheme with numerical citation keys and 
citations appearing in alphabethic order or in the order of citations
is supported as well. This is intended for documents that need
numerical citations due to external demands.}  is used.
The required style files are 
automatically loaded by \texttt{nrdoc} and support both norsk
and english. Some examples are given in Table \ref{tab:cite}.    
\begin{table}[htb]
\centering
\begin{tabular}{lll}
\hline
usual citations &  
\protect\cite{mardia79} & 
\verb!\cite{mardia79}! \\
in parentheses  &  
\protect\citep{mardia79} & 
\verb!\citep{mardia79}!\\
with a see     &  
\protect\citep[see][]{mardia79} &
\verb!\citep[see][]{mardia79}!\\
with a section/page & 
\protect\citet[Section 2]{mardia79} & 
\verb!\citet{mardia79}!\\
will all authors & 
\protect\citet*{mardia79} & 
\verb!\citet*{mardia79}!\\
without parentheses & 
\protect\citealt{mardia79}&
\verb!\citealt{mardia79}!\\
only the year & 
\protect\citeyear{mardia79} & 
\verb!\citeyear{mardia79}!\\
only the author & 
\protect\citeauthor{mardia79} &
\verb!\citeauthor{mardia79}! \\
\hline
\end{tabular}
\caption{Examples of citations using {Bib}\TeX{}.}
\label{tab:cite} 
\end{table}

The citations are listed in a {Bib}\TeX{} file with the extension
\texttt{.bib}. An example (\texttt{ref.bib}) is given in Appendix
\ref{sec:ref.bib}. Use the \texttt{bibtex} command to run {Bib}\TeX{}.


\section{Languages}

The following options are available for 
different language\index{language} support:
\begin{description}
\item[\textit{american}] US english (default).\index{option!american} 
\item[\textit{british}] UK english. \index{option!british} 
\item[\textit{english}] US english. \index{option!english} 
\item[\textit{norsk}] Bokm\aa l.\index{option!norsk} 
\item[\textit{nynorsk}] Nynorsk.\index{option!nynorsk} 
\item[\textit{UKenglish}] UK english. \index{option!UKenglish}
\item[\textit{USenglish}] US english. \index{option!USenglish} 
\end{description}

\section{Additional options}
\subsection{The utf8 option}\index{option!utf8}

Allows use of utf8 font encoding. The current 
default is latin1.

\subsection{The draft option}\index{option!draft}

The draft option prevents figure inclusion, turns off the cover
generation and adds marks on oversized lines. This option also 
prints equation labels in the margin.

\subsection{The showlabels option}\index{option!showlabels}

This option prints equation labels in the margin.

\subsection{The indentpar option}\index{option!indentpar}

This option turns left intendation on for paragraphs.

\subsection{The nodesign option}\index{option!nodesign}

The nodesign option prevents inclusion of the design elements created
by Much Design. It is used to speed up processing in the writing stage.
This option must be removed in the final run.

\subsection{The dvips option}\index{option!dvips}

This package is designed to produce pdf output either via
\LaTeX\ followed by dvipdfm or directly by pdf\LaTeX.
By using the dvips option you can use \LaTeX\ followed by dvips to produce
postscript.

\subsection{The englishTitle option}\index{option!englishTitle}

When specifying both english and norsk as options for multi-lingual
documents the nrdoc style will choose norwegian as language for cover page
and title matter. The option \texttt{englishTitle} will choose english
as default language, including cover and title matter. Switching to
norwegian language is done by \texttt{\textbackslash
  selectlanguage\textbraceleft{norsk}\textbraceright}.


\section{Making an index}\index{index}

There is support for making indexes with makeindex.
Add the command \verb!\makeindex! in the preamble, and
run \texttt{makeindex} on the main tex file. This generates
an index file.


\section{Default packages}

The following packages are loaded by default (or by use of options):
\begin{description}
\item[\textit{a4wide}] For page geometry. \index{package!a4wide} 
\item[\textit{afterpage}] For float control. \index{package!afterpage}
\item[\textit{amsmath}] For advanced mathematical typesetting. \index{package!amsmath}\item[\textit{algoritm}] For algorithm (type of float). 
\index{package!algorithm}
\item[\textit{array}] For added tabular functionality \index{package!array}
\item[\textit{babel}] For language support. \index{package!babel} 
\item[\textit{calc}] For numerical programming constructs in \LaTeX. 
\index{package!calc}
\item[\textit{caption}] For formatting of captions.
\index{package!caption} 
\item[\textit{color}] For colour fonts. \index{package!color}
\item[\textit{fancyhdr}] For headers. \index{package!fancyhdr} 
\item[\textit{fontenc}] For font encoding (using T1). \index{package!fontenc}
\item[\textit{float}] For float control. \index{package!float} 
\item[\textit{graphicx}] for graphics inclusion. \index{package!graphicx} 
\item[\textit{hyperref}] For hyperlinks. \index{package!hyperref}
\item[\textit{ifpdf}] Provides tests for  pdf\LaTeX. 
\index{package!ifpdf} 
\item[\textit{ifthen}] Provides programming constructs in \LaTeX. \index{package!ifthen} 
\item[\textit{indentfirst}] Handles Norwegian type indentation.\index{package!firstindent} 
\item[\textit{inputenc}] Handles Norwegian
  letters. \index{package!inputenc} 
\item[\textit{makeidx}] For index. \index{package!makeidx}
\item[\textit{multicol}] For multiple columns. \index{package!multicol} 
\item[\textit{natbib}] For bibliography. \index{package!natbib} 
\item[\textit{paralist}] For list environments. \index{package!paralist}
\item[\textit{parskip}] For non-indenting paragraphs. \index{package!parskip} 
\item[\textit{rotating}] For text rotation. \index{package!rotating} 
\item[\textit{showlabels}] For printing of labels (in drafts). 
\index{package!showlabels} 
\item[\textit{textcase}] For text lower case and upper case functions.
\item[\textit{textcomp}] For special characters. 
\index{package!textcase}
\item[\textit{textpos}] For absolute positioning of text on front page.
\index{package!textpos} 
\item[\textit{thumbpdf}] For thumbnails in pdf.
\index{package!thumbpdf}
\item[\textit{titlesec}] For title formatting.
\index{package!titlesec} 
\item[\textit{geometry}] For page geometry. \index{package!geometry}


\item[\textit{verbatim}] For verbatim text. \index{package!verbatim} 
Also contains a useful \texttt{comment} environment.

\item[\textit{xspace}] For space control. \index{package!xspace}
\end{description} 
For more information about most of these packages,
consult the book by \cite{latexcompanion}.

\section{Shortcomings}

Pdf bookmarks does not work correctly for chapters and sections in appendices.
Some postprocessing of the bookmark file (file extension .out) 
may be needed to translate \LaTeX\ codes, since bookmarks must 
be written in PDFEncoding. To aid this process, the .out file 
is not rewritten by \LaTeX if it is edited to contain 
a line \let\WriteBookmarks\relax.

\bibliography{ref}

\clearpage
\appendix

\section{Setting Unix variables}
For the system to find the \texttt{nrdoc} class, the variables
\texttt{TEXINPUTS}\index{texinputs@TEXINPUTS} and 
\texttt{BSTINPUTS}\index{texinputs@BSTINPUTS} needs to be set. These
variable should have been set for you, but if you get the message 
\texttt{TEXINPUTS: Undefined variable.} after typing \texttt{echo
\$TEXINPUTS} in Unix/Linux, you need to set the variable:
If you use \texttt{tcsh} (that is
you have a \texttt{.cshrc} file), the  \texttt{.cshrc} file should
include this line:
\begin{verbatim}
setenv TEXINPUTS .:/nr/group/maler/nrdoc:
\end{verbatim}
If you have problems with {Bib}\TeX, it might help to set the variable
\texttt{BSTINPUTS}:
\begin{verbatim}
setenv BSTINPUTS .:/nr/group/maler/nrdoc:
\end{verbatim}

\section{The \LaTeX{} code for this document\label{sec:code}}
\verbatiminput{manual.tex}
\section{The {Bib}\TeX{} file (\texttt{ref.bib})}\label{sec:ref.bib}
\verbatiminput{ref.bib}

\end{document}
